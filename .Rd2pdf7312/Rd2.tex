\documentclass[a4paper]{book}
\usepackage[times,inconsolata,hyper]{Rd}
\usepackage{makeidx}
\usepackage[utf8]{inputenc} % @SET ENCODING@
% \usepackage{graphicx} % @USE GRAPHICX@
\makeindex{}
\begin{document}
\chapter*{}
\begin{center}
{\textbf{\huge gbif.range}}
\par\bigskip{\large \today}
\end{center}
\inputencoding{utf8}
\ifthenelse{\boolean{Rd@use@hyper}}{\hypersetup{pdftitle = {gbif.range: A toolbox to efficiently download and filter large GBIF observational datasets for sound spatial analyses}}}{}\begin{description}
\raggedright{}
\item[Type]\AsIs{Package}
\item[Title]\AsIs{A toolbox to efficiently download and filter large GBIF observational datasets for sound spatial analyses}
\item[Version]\AsIs{0.2.0}
\item[Depends]\AsIs{R (>= 4.0.0), raster, terra,  rgbif, CoordinateCleaner, rgeos, mclust, ClusterR, FNN, geometry}
\item[Description]\AsIs{Workflow to generate taxa range maps from scratch based on ecoregions and an user-friendly GBIF wrapper (no hard-limit of < 100,000 species observations, filetring parameters to easily flag records, access to the GBIF backbone taxonomy)}
\item[License]\AsIs{GPL (>=3)}
\item[BugReports]\AsIs{}\url{https://github.com/8Ginette8/gbif.range/issues}\AsIs{}
\item[Encoding]\AsIs{UTF-8}
\item[Maintainer]\AsIs{Yohann Chauvier }\email{yohann.chauvier@wsl.ch}\AsIs{}
\item[LazyData]\AsIs{true}
\item[RoxygenNote]\AsIs{7.1.2}
\item[Authors]\AsIs{Yohann Chauvier [cre,aut] (https://orcid.org/0000-0001-9399-3192)
Patrice Descombes [aut] (https://orcid.org/0000-0002-3760-9907)
Oskar Hagen [aut] (https://orcid.org/0000-0002-7931-6571)
Camille Albouy [aut] (https://orcid.org/0000-0003-1629-2389)
Fabian Fopp [aut] (https://orcid.org/0000-0003-0648-8484)
Michael P. Nobis [aut] (https://orcid.org/0000-0003-3285-1590)
Philipp Brun [aut] (https://orcid.org/0000-0002-2750-9793)
Lisha Lyu [aut] (https://orcid.org/0000-0001-7855-8109)
Loïc Pellissier [aut] (https://orcid.org/0000-0002-2289-8259)
Katalin Csillery [aut] (https://orcid.org/0000-0003-0039-9296)}
\item[Collate]\AsIs{'make_tiles.R'
'get_gbif.R'
'get_taxonomy.R'
'get_doi.R'
'obs_filter.R'
'get_range.R'
'conv_function.R'}
\end{description}
\Rdcontents{\R{} topics documented:}
\inputencoding{utf8}
\HeaderA{conv\_function}{Create polygon objects in different bioregions}{conv.Rul.function}
%
\begin{Description}\relax
Not to be called directly by the user
\end{Description}
%
\begin{Usage}
\begin{verbatim}
conv_function(sp_coord, bwp, bipl, bwpo, temp_dir, g = NULL)
\end{verbatim}
\end{Usage}
\inputencoding{utf8}
\HeaderA{get\_doi}{Get a custom DOI for a GBIF filtered dataset}{get.Rul.doi}
%
\begin{Description}\relax
A small user friendly wrapper of the derived\_dataset() function of the
rgbif R package, compatible with one or several get\_gbif() outputs.
\end{Description}
%
\begin{Usage}
\begin{verbatim}
get_doi(
  get.gbif = NULL,
  title = NULL,
  description = NULL,
  source_url = "https://example.com/",
  user = "",
  pwd = "",
  ...
)
\end{verbatim}
\end{Usage}
%
\begin{Arguments}
\begin{ldescription}
\item[\code{get.gbif}] data.frame or list. One get\_gbif() output or a list of several.

\item[\code{title}] The title for your derived dataset.

\item[\code{source\_url}] A link to where the dataset is stored.

\item[\code{user}] Your GBIF username.

\item[\code{pwd}] Your GBIF password.

\item[\code{...}] Additonnal parameters for derived\_dataset() in rgbif.

\item[\code{descritpion}] A description of the dataset.
\end{ldescription}
\end{Arguments}
%
\begin{Details}\relax
see derived\_dataset() function from the rgbif R package
\end{Details}
%
\begin{Value}
One citable DOI and its information.
\end{Value}
%
\begin{References}\relax
Chamberlain, S., Oldoni, D., \& Waller, J. (2022). rgbif: interface to the global biodiversity
information facility API. 10.5281/zenodo.6023735
\end{References}
%
\begin{SeeAlso}\relax
The rgbif package for additional and more general approaches to get GBIF DOI
\end{SeeAlso}
%
\begin{Examples}
\begin{ExampleCode}

# Downloading worldwide the observations of Panthera tigris and Ailuropoda melanoleuca
obs.pt = get_gbif("Panthera tigris")
obs.am = get_gbif("Ailuropoda melanoleuca")

# Just an example on how to retrieve the DOI for only one get_gbif() output
get_doi(obs.pt,title="GBIF_test1",description="A small example 1",
    source_url="https://example.com/",user="",pwd="") # Use your own GBIF credentials here

# Just an example on how to retrieve the DOI for several get_gbif() outputs
get_doi(list(obs.pt,obs.am),title="GBIF_test2",description="A small example 2",
    source_url="https://example.com/",user="",pwd="") # Use your own GBIF credentials here

\end{ExampleCode}
\end{Examples}
\inputencoding{utf8}
\HeaderA{get\_gbif}{Massively download and filter GBIF observations for sound spatial analyses}{get.Rul.gbif}
%
\begin{Description}\relax
Implement an user-friendly workflow to download and clean gbif taxa observations.
The function uses the rgbif R package but (1) implements the same search result 
found if www.gbif.org is employed i.e., based on the input taxa name, all species
records related to its accepted name and synonyms are extracted. The function
also (2) bypasses rgbif hard limit on the number of records (100'000 max).
For this purpose, a dynamic moving window is created and used across the geographic
extent defined by the user. This window automatically fragments the specified
study area in succesive tiles of different sizes, until all tiles include < 100'000
observations. The function also (3) automatically applies a post-filtering of
observations based on the chosen resolution of the study/analysis and by partly
employing the CoordinateCleaner R package. Filtering options may be chosen and
involve several choices: study's extent, removal of duplicates, removal of absences,
basis of records selection, removal of invalid/uncertain xy coordinates (WGS84), time
period selection and removal of raster centroids. By default, the argument
hasGeospatialIssue in occ\_search() (implemented rgbif function) is set to FALSE.
\end{Description}
%
\begin{Usage}
\begin{verbatim}
get_gbif(
  sp_name = NULL,
  conf_match = 90,
  geo = NULL,
  grain = 1000,
  duplicates = FALSE,
  absences = FALSE,
  no_xy = FALSE,
  basis = c("OBSERVATION", "HUMAN_OBSERVATION", "MACHINE_OBSERVATION",
    "MATERIAL_SAMPLE", "PRESERVED_SPECIMEN", "FOSSIL_SPECIMEN", "LIVING_SPECIMEN",
    "LITERATURE", "UNKNOWN"),
  add_infos = NULL,
  time_period = c(1000, 3000),
  identic_xy = FALSE,
  wConverted_xy = FALSE,
  centroids = FALSE,
  ntries = 10,
  error.skip = TRUE,
  ...
)
\end{verbatim}
\end{Usage}
%
\begin{Arguments}
\begin{ldescription}
\item[\code{sp\_name}] Character. Scientific name to run an online search
(i.e. with GBIF-API) for species observations. Works also for genus and higher taxa
levels.

\item[\code{conf\_match}] Numeric from 0 to 100. Determine the confidence threshold of match
of 'sp\_name' with the GBIF backbone taxonomy. Default is 90.

\item[\code{geo}] Object of class Extent, SpatExtent, SpatialPolygon, SpatialPolygonDataframe,
or SpaVector (WGS84) to define the study's area extent. Default is NULL i.e. the whole globe.

\item[\code{grain}] Numeric. Specify in meters the study resolution. Used to
filter gbif records (x2) according to their uncertainties and number of coordinate
decimals. Records with no information on coordinate uncertainties (column
'coordinateUncertaintyInMeters') are be kept by default. See details.

\item[\code{duplicates}] Logical. Should duplicated records be kept?

\item[\code{absences}] Logical. Should absence records be kept?

\item[\code{no\_xy}] Logical. Default is FALSE i.e. only records with coordinates are
downloaded. If TRUE, only records with no coordinates are downloaded.

\item[\code{basis}] Character. Which basis of records should be selected?
Default is all i.e. c('OBSERVATION', 'HUMAN\_OBSERVATION', 'MACHINE\_OBSERVATION',
'MATERIAL\_SAMPLE', 'PRESERVED\_SPECIMEN', 'FOSSIL\_SPECIMEN', 'LIVING\_SPECIMEN', 'LITERATURE',
'UNKNOWN'). Description may be found here: https://docs.gbif.org/course-data-use/en/basis-of-record.html

\item[\code{add\_infos}] Character. Infos that may be added to the default output information.
List of IDs may be found at: https://www.gbif.org/developer/occurrence.
Default IDs contain 'taxonKey', 'scientificName', 'acceptedTaxonKey',
'acceptedScientificName', 'individualCount', 'decimalLatitude', 'decimalLongitude',
'basisOfRecord', 'coordinateUncertaintyInMeters', 'country', 'year', 'datasetKey', 
'institutionCode', 'publishingOrgKey', 'taxonomicStatus' and 'taxonRank'.

\item[\code{time\_period}] Numerical vector. Observations will be downloaded according to the chosen
year range. Default is c(1000,3000). Observations with year = NA are kept by default.

\item[\code{identic\_xy}] Logical. Should records with identical xy be kept?

\item[\code{wConverted\_xy}] Logical. Should incorrectly lon/lat converted xy be kept?
Uses cd\_ddmm() from 'CoordinateCleaner' R package.

\item[\code{centroids}] Logical. Should species records from raster centroids be kept?
Uses cd\_round() from 'CoordinateCleaner' R package.

\item[\code{ntries}] Numeric. In case of failure from GBIF server or within the rgbif package, how many
download attempts should the function request? Default is '10' with a 2 seconds interval
between tries. If attempts failed, an empty data.frame is return by default.

\item[\code{error.skip}] Logical. Should the search process continues if ntries failed ?

\item[\code{...}] Additonnal parameters for the function cd\_round() of CoordinateCleaner.
\end{ldescription}
\end{Arguments}
%
\begin{Details}\relax
Argument `grain` used for two distinct gbif records filtering. (1) Records filtering
according to gbif 'coordinateUncertaintyInMeters'; every records uncertainty > grain/2
are removed. Note: Records with no information on coordinate uncertainties are kept by
default. (2) Records filtering according to the number of longitude/latitude decimals;
if 110km < grain <= 11km, lon/lat with >= 1 decimal are kept, if 11km < grain <= 1100m,
lon/lat with >= 2 decimals kept; if 1100m < grain <= 110m, lon/lat with >= 3 decimals
are kept; if 110m < grain <= 11m, lon/lat with >= 4 decimals are kept;
if 11m < grain <= 1.1m, lon/lat with >= 5 decimals are kept etc...
\end{Details}
%
\begin{Value}
Object of class data.frame with requested GBIF information. Although the function
works accurately, error outputs might still occur depending on the 'sp\_name' used.
Therefore, default information detailed in 'add\_infos' is stored so that sanity checks
may still be applied afterwards. Although crucial preliminary checks of species records
are done by the function, addtional post exploration with the CoordinateCleaner R
package is still highly recommended.
\end{Value}
%
\begin{References}\relax
Chauvier, Y., Thuiller, W., Brun, P., Lavergne, S., Descombes, P., Karger, D. N., ... \& Zimmermann,
N. E. (2021). Influence of climate, soil, and land cover on plant species distribution in the
European Alps. Ecological monographs, 91(2), e01433. 10.1002/ecm.1433

Chamberlain, S., Oldoni, D., \& Waller, J. (2022). rgbif: interface to the global biodiversity
information facility API. 10.5281/zenodo.6023735

Zizka, A., Silvestro, D., Andermann, T., Azevedo, J., Duarte Ritter, C., Edler, D., ... \& Antonelli,
A. (2019). CoordinateCleaner: Standardized cleaning of occurrence records from biological collection
databases. Methods in Ecology and Evolution, 10(5), 744-751. 10.1111/2041-210X.13152

Hijmans, Robert J. "terra: Spatial Data Analysis. R Package Version 1.6-7." (2022). Terra - CRAN
\end{References}
%
\begin{SeeAlso}\relax
The (1) rgbif and (2) CoordinateCelaner packages for additional and more general
approaches on (1) downloading GBIF observations and (2) post-filetering those.
\end{SeeAlso}
%
\begin{Examples}
\begin{ExampleCode}

# Load maptools for the map world
library(maptools)
data(wrld_simpl)

# Load the Alps Extend
data(geo_dat)

# Downloading worldwide the observations of Panthera tigris
obs.pt = get_gbif("Panthera tigris",basis=c("OBSERVATION","HUMAN_OBSERVATION"))
plot(wrld_simpl)
points(obs.pt[,c("decimalLongitude","decimalLatitude")],pch=20,col="#238b4550",cex=4)

# Downloading in the Alps the observations of Cypripedium calceolus (with a 100m grain and
# by adding the 'issues' column)
obs.cc = get_gbif("Cypripedium calceolus", geo = shp.lonlat, grain = 100, add_infos = c("issue"))
plot(shp.lonlat)
points(obs.cc[,c("decimalLongitude","decimalLatitude")],pch=20,col="#238b4550",cex=1)

# Downloading worlwide the observations of Ailuropoda melanoleuca (with a 100km grain, after 1990
# and by keeping duplicates and by adding the name of the person who collected the panda records)
obs.am = get_gbif("Ailuropoda melanoleuca", grain = 100000 , duplicates = TRUE,
    time_period = c(1990,3000), add_infos = c("recordedBy","issue"))
plot(wrld_simpl)
points(obs.am[,c("decimalLongitude","decimalLatitude")],pch=20,col="#238b4550",cex=4)

# Downloading worlwide the observations of Phascolarctos cinereus (with a 1km grain, after 1980,
# and keeping raster centroids)
obs.pc = get_gbif("Phascolarctos cinereus", grain = 1000,
    time_period = c(1990,3000), centroids = TRUE)

\end{ExampleCode}
\end{Examples}
\inputencoding{utf8}
\HeaderA{get\_range}{Create a species range map based on a get\_gbif() output}{get.Rul.range}
%
\begin{Description}\relax
Estimates species ranges based on occurrence data (GBIF or not) and bioregions.
It first deletes outliers from the observation dataset and then creates a polygon
(convex hull) with a user specified buffer around all the observations of one bioregion.
If there is only one observation in a bioregion, a buffer around this point
will be created. If all points in a bioregion are on a line, the function will also
create a buffer around these points, however, the buffer size increases with the number
of points in the line.
\end{Description}
%
\begin{Usage}
\begin{verbatim}
get_range(
  sp_name = NULL,
  occ_coord = NULL,
  Bioreg = eco.earth,
  Bioreg_name = "ECO_NAME",
  degrees_outlier = 3,
  clustered_points_outlier = 2,
  buffer_width_point = 4,
  buffer_increment_point_line = 0.5,
  buffer_width_polygon = 4,
  dir_temp = paste0("temp", sample(1:99999999, 1)),
  raster = TRUE,
  res = 10
)
\end{verbatim}
\end{Usage}
%
\begin{Arguments}
\begin{ldescription}
\item[\code{sp\_name}] Character of the species name. E.g. 'Anemone nemorosa'.

\item[\code{occ\_coord}] get\_gbif() output or SpatialPoints object.

\item[\code{Bioreg}] SpatialPolygonsDataFrame containg different bioregions (convex hulls will
be classified on a bioreg basis). Although whatever shapefile may be set as input, note
that three ecoregion shapefiles are already included in the library: eco.earh' (for
terrestrial species; Nature conservancy version adapted from Olson \& al. 2001), eco.marine'
(for coastal and reef species; Spalding \& al. 2007) and 'eco.fresh' (for freshwater species;
Abell \& al. 2008). For deep ocean/sea species, 'eco.earth' may be used, but the polygon
estimates will only be geographic. Default is 'eco.earth'.

\item[\code{Bioreg\_name}] How is the slot containing the bioregion names called? Default is the
very detailed level of 'eco.earth' or "ECO\_NAME".

\item[\code{degrees\_outlier}] distance threshold (degrees) for outlier classification. If the
nearest minimal distance to the next point is larger than this threshold, it will be
considered as an outlier.

\item[\code{clustered\_points\_outlier}] maximum number of points which are closer to each other
than the degrees\_outlier, but should still be considered as outliers.

\item[\code{buffer\_width\_point}] buffer (in degrees) which will be applied around single observations.

\item[\code{buffer\_increment\_point\_line}] how much should the buffer be increased for each point on a line.

\item[\code{buffer\_width\_polygon}] buffer (in degrees) which will be applied around distribution polygons
(for each bioregion).

\item[\code{dir\_temp}] where should the temporary text file for the convex hull be saved?
(text file will be deleted again).

\item[\code{raster}] Logical. Should the output be a unified raster? Default is TRUE

\item[\code{res}] Numeric. If raster = TRUE, which resolution? Final resolution in ° = 1°/res
e.g.,  = 0.1° if res = 10. Default is 10.
\end{ldescription}
\end{Arguments}
%
\begin{Details}\relax
...
\end{Details}
%
\begin{Value}
A Shapefile or a SpatRaster
\end{Value}
%
\begin{References}\relax
Lyu, L., Leugger, F., Hagen, O., Fopp, F., Boschman, L. M., Strijk, J. S., ... \&
Pellissier, L. (2022). An integrated high resolution mapping shows congruent biodiversity
patterns of Fagales and Pinales. New Phytologist, 235(2), 759-772 10.1111/nph.18158

Olson, D. M., Dinerstein, E., Wikramanayake, E. D., Burgess, N. D., Powell, G. V. N.,
Underwood, E. C., D'Amico, J. A., Itoua, I., Strand, H. E., Morrison, J. C., Loucks, C. J.,
Allnutt, T. F., Ricketts, T. H., Kura, Y., Lamoreux, J. F., Wettengel, W. W., Hedao, P., Kassem,
K. R. 2001. Terrestrial ecoregions of the world: a new map of life on Earth.
Bioscience 51(11):933-938. doi: 10.1641/0006-3568(2001)051

Mark D. Spalding, Helen E. Fox, Gerald R. Allen, Nick Davidson, Zach A. Ferdaña, Max
Finlayson, Benjamin S. Halpern, Miguel A. Jorge, Al Lombana, Sara A. Lourie, Kirsten D.
Martin, Edmund McManus, Jennifer Molnar, Cheri A. Recchia, James Robertson, Marine
Ecoregions of the World: A Bioregionalization of Coastal and Shelf Areas, BioScience,
Volume 57, Issue 7, July 2007, Pages 573–583. doi: 10.1641/B570707

Robin Abell, Michele L. Thieme, Carmen Revenga, Mark Bryer, Maurice Kottelat, Nina Bogutskaya,
Brian Coad, Nick Mandrak, Salvador Contreras Balderas, William Bussing, Melanie L. J. Stiassny,
Paul Skelton, Gerald R. Allen, Peter Unmack, Alexander Naseka, Rebecca Ng, Nikolai Sindorf,
James Robertson, Eric Armijo, Jonathan V. Higgins, Thomas J. Heibel, Eric Wikramanayake,
David Olson, Hugo L. López, Roberto E. Reis, John G. Lundberg, Mark H. Sabaj Pérez,
Paulo Petry, Freshwater Ecoregions of the World: A New Map of Biogeographic Units for
Freshwater Biodiversity Conservation, BioScience, Volume 58, Issue 5, May 2008,
Pages 403–414. doi: 10.1641/B580507

Hijmans, Robert J. "terra: Spatial Data Analysis. R Package Version 1.6-7." (2022). Terra - CRAN
\end{References}
%
\begin{SeeAlso}\relax
...
\end{SeeAlso}
%
\begin{Examples}
\begin{ExampleCode}
# Load available ecoregions
data(ecoregions)

# First download the worldwide observations of Panthera tigris and convert to SpatialPoints
obs.pt = get_gbif("Panthera tigris",basis=c("OBSERVATION","HUMAN_OBSERVATION"))

# Plot
plot(eco.earth)
plot(sp.shp,pch=20,col="#238b4550",cex=4,add=TRUE)

# Generate the distributional range map of Panthera tigris for the finest terrestrial ecoregions
range.tiger = get_range("Panthera tigris",sp.shp,eco.earth,"ECO_NAME")

# Plotting
plot(eco.earth)
plot(range.tiger,col="#238b45",add=TRUE)

\end{ExampleCode}
\end{Examples}
\inputencoding{utf8}
\HeaderA{get\_taxonomy}{Retrieve from GBIF all scientific names of a specific Taxa}{get.Rul.taxonomy}
%
\begin{Description}\relax
Generates, based on a given species name, a list of all its scientific names
(accepted, synonyms) found in the GBIF backbone taxonomy to download the data.
Children and related doubtful names not use to download the data may also be extracted.
\end{Description}
%
\begin{Usage}
\begin{verbatim}
get_taxonomy(sp_name = NULL, conf_match = 90, all = FALSE)
\end{verbatim}
\end{Usage}
%
\begin{Arguments}
\begin{ldescription}
\item[\code{sp\_name}] Character. Species name from which the user wants to retrieve all existing GBIF names

\item[\code{conf\_match}] Numeric. From 0 to 100. Determine the confidence
threshold of match of 'sp\_name' with the GBIF backbone taxonomy. Default is 90.

\item[\code{all}] Logical. Default is FALSE. Should all species names be retrieved or only
the accepted name and its synonyms?
\end{ldescription}
\end{Arguments}
%
\begin{Value}
A data.frame with two columns: (1) Names and (2) Backbone Taxonomy Status
\end{Value}
%
\begin{References}\relax
Chamberlain, S., Oldoni, D., \& Waller, J. (2022). rgbif: interface to the global biodiversity
information facility API. 10.5281/zenodo.6023735
\end{References}
%
\begin{SeeAlso}\relax
The rgbif package for additional and more general approaches on how to retrieve
scientific names from the GBIF backbone taxonomy.
\end{SeeAlso}
%
\begin{Examples}
\begin{ExampleCode}
get_taxonomy("Cypripedium calceolus",all=FALSE)
get_taxonomy("Cypripedium calceolus",all=TRUE)

\end{ExampleCode}
\end{Examples}
\inputencoding{utf8}
\HeaderA{make\_tiles}{Create a specific number of tiles based on a raster extent}{make.Rul.tiles}
%
\begin{Description}\relax
Based on a specific extent, one or several tiles are generated. Tiles can be smaller
raster extents or geometry arguments POLYGON(). The original extent is therefore either
converted into a POLYGON() argument, or divided into Ntiles of regular fragments which are
converted into POLYGON() arguments and smaller SpatExtent.
\end{Description}
%
\begin{Usage}
\begin{verbatim}
make_tiles(geo, Ntiles, sext = TRUE)
\end{verbatim}
\end{Usage}
%
\begin{Arguments}
\begin{ldescription}
\item[\code{geo}] Object of class Extent, SpatExtent, SpatialPolygon, SpatialPolygonDataframe,
or SpaVector (WGS84 or planar) to define the study's area extent. Default is NULL i.e. the
whole globe.

\item[\code{Ntiles}] Numeric. In how many tiles/fragments should geo be divided approximately?

\item[\code{sext}] Logical. Should a list of SpatExtent also be returned for each generated POLYGON()?
\end{ldescription}
\end{Arguments}
%
\begin{Value}
A list of geometry arguments POLYGON() of length Ntiles (and of SpatExtent
if sext=TRUE)
\end{Value}
%
\begin{References}\relax
Chauvier, Y., Thuiller, W., Brun, P., Lavergne, S., Descombes, P., Karger, D. N., ... \& Zimmermann,
N. E. (2021). Influence of climate, soil, and land cover on plant species distribution in the
European Alps. Ecological monographs, 91(2), e01433. 10.1002/ecm.1433
\end{References}
%
\begin{Examples}
\begin{ExampleCode}

# Load the European Alps Extent
data(geo_dat)

# Apply the function to divide the extent in ~20 fragments
mt = make_tiles(geo=shp.lonlat,Ntiles=20,sext=TRUE); mt
\end{ExampleCode}
\end{Examples}
\inputencoding{utf8}
\HeaderA{obs\_filter}{Filter a set of GBIF observations according to a defined grain}{obs.Rul.filter}
%
\begin{Description}\relax
Whereas the 'grain' parameter in get\_gbif() allows GBIF observations to be
filtered according to a certain spatial precision, obs\_filter() accepts
as input a get\_gbif() output (one or several species) and filter the
observations according to a specific given grid resolution (one observation
per pixel grid kept). This function allows the user to refine the density of
GBIF observations according to a defined analysis/study's resolution.
\end{Description}
%
\begin{Usage}
\begin{verbatim}
obs_filter(get.gbif, grid)
\end{verbatim}
\end{Usage}
%
\begin{Arguments}
\begin{ldescription}
\item[\code{get.gbif}] one get\_gbif() output including one or several species. Note
that if GBIF absences are kept in the output(s), the function should be used
distinctively for observations and absences.

\item[\code{grid}] Object of class SpatRaster, RasterLayer, RasterBrick or
RasterStack of desired resolution and extent (WGS84)
\end{ldescription}
\end{Arguments}
%
\begin{Value}
a data frame with two columns named 'x' and 'y' comprising
the new set of observations filtered at grid resolution.
\end{Value}
%
\begin{Examples}
\begin{ExampleCode}

# Load the European Alps extent and a raster of a random resolution
data(geo_dat)
data(exrst)

# Downloading in the European Alps the observations of two plant species
obs.arcto = get_gbif("Arctostaphylos alpinus",geo=shp.lonlat)
obs.saxi = get_gbif("Saxifraga cernua",geo=shp.lonlat)
plot(vect(shp.lonlat))
points(obs.arcto[,c("decimalLongitude","decimalLatitude")],pch=20,col="#238b4550",cex=1)
points(obs.saxi[,c("decimalLongitude","decimalLatitude")],pch=20,col="#99000d50",cex=1)

# rbind both datasets
both.sp = rbind(obs.arcto,obs.saxi)

# Run function
obs.filt = obs_filter(both.sp,rst)

# Check new points
x11();plot(vect(shp.lonlat))
points(obs.filt[obs.filt$Species%in%"Arctostaphylos alpinus",c("x","y")],pch=20,col="#238b4550",cex=1)
points(obs.filt[obs.filt$Species%in%"Saxifraga cernua",c("x","y")],pch=20,col="#99000d50",cex=1)

\end{ExampleCode}
\end{Examples}
\printindex{}
\end{document}
